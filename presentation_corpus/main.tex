%% structure du document et géométrie
\documentclass[hidelinks, 11pt, letterpaper]{article}
\usepackage[margin=1.15in]{geometry}
\usepackage[T1]{fontenc}
% \usepackage{sectsty}
% \allsectionsfont{\singlespacing}
\widowpenalty=9999
\clubpenalty=9999

%% indications linguistiques
\usepackage[french, english]{babel}

%% indications typographiques
\usepackage{lmodern}
\usepackage{seqsplit}
\usepackage{etoolbox}
    \AtBeginEnvironment{quote}{\par\singlespacing\small} % pour un interligne simple dans les blockquotes
\renewcommand{\baselinestretch}{1.5} % l'interligne
\newcommand{\guil}[1]{<<~{#1}~>>} % guillemets français de premier niveau
\newcommand{\guill}[1]{``{#1}''} % guillemets dans les guillemets
\usepackage[normalem]{ulem}

%% gestion bibliographique
\usepackage{csquotes}
\usepackage[style=apa,isbn=false,language=french,backend=biber]{biblatex}
\DefineBibliographyStrings{french}{in={dans},inseries={dans}}
\AtEveryCitekey{\clearfield{pagetotal}} %AtEveryBibitem (bibliographie)
\AtEveryCitekey{\clearfield{pages}} %pages = quand chapitre d'un livre
\AtEveryBibitem{\clearfield{pagetotal}} %pages = quand chapitre d'un livre
\AtEveryBibitem{\clearfield{url}}
\AtEveryBibitem{\clearfield{urldate}}
\DeclareNameAlias{sortname}{last-first}
\DefineBibliographyExtras{french}{\restorecommand\mkbibnamefamily} %pour les noms de famille pas en small caps
\nocite{*}

\addbibresource{partie_1.bib}
% \setcounter{tocdepth}{1} %(3 = jusqu'à subsubsection) *CHECKER CE QUE ÇA FAIT
% \setcounter{secnumdepth}{1} % *CHECKER CE QUE ÇA FAIT

%% design et mise en page
\usepackage{xcolor}
\usepackage{soul}
\usepackage{hyperref} % pour la gestion des hyperliens internes

%% notes de bas de page
\usepackage[hang, flushmargin]{footmisc}

%% test kitchen
\usepackage{blindtext}
\usepackage{lipsum}

\begin{document}
\selectlanguage{french}
\pagenumbering{gobble} % bouffe la numérotation des pages

\begin{titlepage}
    \begin{center}
        M\textsc{c}DUFF, Lydia \hfill
        20294887\\
        SAVARD-ARSENEAULT, Adrien \hfill
        20155684
        \vfill
        \textbf{Définition de la problématique et état de la question}
        \vfill
        Travail présenté au\\
        Pr Dominic FOREST\\
        \vspace*{1.5cm}
        dans le cadre du cours\\
        SCI6203 -- Intelligence artificielle et données textuelles
        \vfill
        École de bibliothéconomie et des sciences de l'information\\
        Université de Montréal\\
        20 octobre 2025
    \end{center}
\end{titlepage}


\pagenumbering{arabic} % {roman} pour les chiffres romains
\setcounter{page}{1}

% Précisions fournies en classe :
% 1) on envoie le corpus tel que trouvé (on ne convertit pas les documents dans d'autres formats);
% 2) on peut au besoin ajouter d'autres critères de constitution non mentionnés en classe;
% 3) on peut se servir d'extraits du corpus pour illustrer nos propos;
% 4) la longueur attendue est d'environ 2 pages;
% 5) si on constitue nous-mêmes notre corpus, M. Forest ne s'attend pas à ce que notre corpus soit finalisé, mais qu'au minium on précise de quelle façon on compte le constituer.

% @Adrien : Devrait-on ajouter une introduction ou reformuler la section des critères généraux pour en faire une introduction ? @Lydia sincèrement je ne sais pas trop... Vu qu'il a l'air de vouloir qu'on aille droit au but j'aurais tendance à laisser tel quel.

\section*{Critères généraux}
Le corpus s'intitule \guil{Littérature critique anglophone autour du \emph{Roman de Silence}}. Il est constitué de documents rassemblés par Adrien Savard-Arseneault depuis l'automne 2020, dont plusieurs qui ont été ajoutés en été 2023, puis en automne 2025.
Il est en cours de conversion et d'annotation par Lydia McDuff et Adrien Savard-Arseneault.
% @Adrien : Pour la date de constitution, est-ce qu'on mentionne que tu avais déjà assembler un corpus avant le cours ? @Lydia : on pourrait ! Ce sont des documents que je rassemble depuis l'automne 2020 environ, avec un biltz à l'été 2023 et à l'automne 2025 @Adrien : j'ai reformulé la phrase en intégrant les dates et après coup je me suis demandée à quoi je servais dans tout ça lol ¯\_( ͡° ͜ʖ ͡°)_/¯ @Lydia oh non :(((((
Ces documents ont été publiés au cours des cinquante dernières années, le plus vieux datant de 1985 et le plus récent, de juillet~2025. Le corpus peut être cité de la façon suivante :

Savard-Arseneault, A. et McDuff, L. (2025). \emph{Littérature critique anglophone autour du \emph{Roman de Silence}} [corpus]. Université de Montréal. \texttt{\url{https://github.com/pitoftar/sci6203a25/tree/main/corpus}}

Constitué de 90~documents scientifiques (articles, chapitres de livres, thèses et mémoires), le corpus a une taille totale de 1~032~363 mots.
Les documents comptent généralement entre 6~000 et 12~000 mots, avec une moyenne de 11~470 mots par document.
La médiane est de 8~463,5 mots.

57 documents sont accessibles en ligne grâce à une licence institutionnelle, 20 documents sont en accès ouvert en ligne (dont 3 avec un embargo), 8 documents sont disponibles à la Bibliothèque des lettres et sciences humaines de l'Université de Montréal, 4 documents sont sous licence CC BY-NC-ND 4.0 et 1 sous licence CC BY-SA 4.0. La majorité des documents constituant le corpus ne sont donc pas complètement libres de droits.

\section*{Critères technologiques}
\label{sec:crittech}
Le corpus a été constitué en fouillant les bases de données pertinentes\footnote{Les bases de données interrogées sont les suivantes: JSTOR, Project Muse, ProQuest, Web of Science, Google Scholar, ainsi que le catalogue des documents disponibles dans les bibliothèques de l'Université de Montréal \emph{via} WorldCat.} au sujet à l'aide de la requête \texttt{"roman de silence" OR "romance of silence"}.
L'outil de recherche ne permettait pas toujours les requêtes combinées: dans ces cas, les deux termes de recherche ont été indiqués séparément.
Lorsque cela était possible, des filtres ont également été appliqués à la recherche afin de ne retenir que les documents en langue anglaise.
Les textes ont été individuellement survolés afin de s'assurer de leur pertinence avec le sujet avant d'être ajoutés au corpus.

En raison de sa nature, la majorité du corpus (83\%, soit 75~documents sur 90) a été récupérée au format PDF.
De ce nombre, un peu moins de la moitié des documents sont nativement numériques alors que 39 sont issus de la numérisation.
Le texte est alors mis en forme à l'aide de procédés de reconnaissance optique des caractères (ROC).
Le reste du corpus (15~documents) a été récupéré au format HTML ou texte brut (\texttt{.txt}).
Nous expérimentons actuellement avec plusieurs manières pour extraire le texte des PDF afin de les baliser dans un format plus structuré et exploitable (HTML ou XML).

Certaines données du corpus (langue, longueur des textes, provenance, année, nom de l'autrice\footnote{Le féminin a été choisi pour alléger le texte et proposer une alternative au masculin comme genre neutre.}) ont été annotés manuellement par les membres de l'équipe à l'aide du logiciel Zotero.
Les annotations de cette classe ont été traduites et structurées au format JSON.

% @Adrien : si je me fie aux notes de cours 2 sur Studium, "support" ferait référence au support sur lequel se trouve le corpus; dans notre cas, on mentionnerait qu'il est rassemblé sur Github ? Aussi, pour les annotations, est-qu'on mentionne aussi les droits/licences et le genre de l'autrice ?

\section*{Critères informationnels}
Tous les textes composant le corpus sont liés aux disciplines de la littérature, de la philologie ou des études médiévales.
Certains intègrent aussi des approches d'études féministes ou de codicologie.
Ils partagent tous un sujet, soit l'étude du \emph{Roman de Silence}, un texte médiéval du XII\textsuperscript{e}~siècle.
Puisque la tâche que nous envisageons réaliser est l'analyse linguistique de la désignation du personnage principal, prénommé Silence, dans le corpus critique, il était nécessaire de rassembler le plus grand nombre de textes possibles traitant de ce sujet.

\section*{Critères linguistiques}
Le corpus est constitué de 27~chapitres de livres, une retranscription d'une communication de colloque, 54~articles de revues scientifiques évalués par les pairs et 8~thèses ou mémoires.
Le registre de langue est soutenu, et comprend quelques termes spécialisés.
Tous les textes sont rédigés en langue anglaise.
Certains d'entre eux comprennent toutefois des citations issues du texte original, en ancien français.
Celles-ci sont généralement indiquées entre guillemets (\guill{}) ou placées dans des blocs de citation à l'écart du texte.

\section*{Difficultés rencontrées et commentaires}
Puisqu'à notre connaissance aucun corpus sur la littérature critique anglophone autour du \emph{Roman de Silence} n'avait été constitué auparavant, nous avons dû le faire nous-mêmes. Cela dit, le repérage de documents pertinents n'a pas été un défi en soi : les difficultés rencontrées résident plutôt dans l'annotation et la transformation des documents. 

Afin de pouvoir comparer la désignation de Silence dans les textes critiques avec les genres des autrices, nous avons annoté chaque document dans Zotero en ce sens. Pour ce faire, certains des documents comprenaient une biographie de l'autrice, mais pour la grande majorité nous avons dû rechercher le nom de l'autrice sur un moteur de recherche\footnote{Par souci de rigueur et de respect envers les autrices, nous ne voulions pas présumer leur genre selon leur prénom.}.
Cette partie de la recherche, bien que nécessaire, s'est révélée chronophage.

Par ailleurs, tel que mentionné dans les \hyperref[sec:crittech]{critères technologiques}, la majorité des documents n'étant disponibles que sous format PDF.
Nous devrons donc trouver une façon efficace de convertir leur texte sous un format davantage approprié à la fouille de texte et qui, nous l'espérons, ne nécessitera peu de correction manuelle de notre part.

% @Adrien : devrait-on ajouter une note comme dans le premier travail pour préciser que le féminin a été employé pour alléger le texte ? @Lydia: ah, pas une mauvaise idée!!

\end{document}
